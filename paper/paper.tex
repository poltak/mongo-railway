\documentclass[a4paper,11pt]{article}

\usepackage[paper=a4paper,left=30mm,width=150mm,top=25mm,bottom=25mm]{geometry}
\usepackage[hidelinks]{hyperref}    % All references will auto hyperlink
\usepackage[font={small}]{caption}  % Make figure captions smaller
\usepackage{graphicx}               % For the displaying of images in floats
\usepackage{mathtools}              % For more advanced math usage
\usepackage{poltakmacros}           % Personal macro package (https://gist.github.com/poltak/6ed59e8f30c9c399143f)



\begin{document}

\begin{titlepage}
\begin{center}


\includegraphics[width=0.3\textwidth]{img/MonashCrest.pdf}~\\[1cm]

\textsc{\LARGE Monash University}\\[1.5cm]
\textsc{\LARGE Faculty of Information Technology}\\[1cm]
\textsc{\Large Computer Science Honours Reading Unit}\\[0.5cm]         % TODO

% Title
\rule{\linewidth}{0.5mm} \\[0.4cm]
{ \huge \bfseries Case Study of Inconsistent Railway Data With MongoDB \\[0.4cm] }   % TODO

\rule{\linewidth}{0.5mm} \\[1.5cm]

% Author and supervisor
\noindent
\begin{minipage}{0.4\textwidth}
\begin{flushleft} \large
\emph{Author:}\\
Jonathan Poltak \textsc{Samosir}
\end{flushleft}
\end{minipage}%
\begin{minipage}{0.4\textwidth}
\begin{flushright} \large
\emph{Supervisor:} \\
Dr.~Maria \textsc{Indrawan-Santiago}
\end{flushright}
\end{minipage}

\vfill

% Bottom of the page
{\large \today}

\end{center}
\end{titlepage}                 % Pull in title page

\newpage
\pagenumbering{roman}
\tableofcontents
\newpage

\pagenumbering{arabic}


\section{Introduction} % (fold)
\label{sec:introduction}

While relational database management systems (RDBMS) have been somewhat of a ``go-to'' solution for for a number of years
for general data storage and management problems that many application developers face, we have noted a recent rise
in the use of non-relational data management tools~\cite{padhy2011rdbms}. Most of these tools have traditionally fallen into the domain of
big data analytics, with platforms such as the Hadoop ecosystem~\footnote{https://hadoop.apache.org/} being notably popular.
However, outside of the domain of big data, looking more at general purpose data storage and management, what are now
commonly referred to as ``NoSQL'' solutions are proving to be a popular solution.

NoSQL databases refer to those databases that are not built on top of the relational algebraic concepts, as laid out by
Codd in 1970~\cite{codd1970relational}, unlike the more commonly used RDBMS technologies, such as
MySQL~\footnote{https://www.mysql.com/}. Being free of the strictness the relational model enforces on its data allows
NoSQL databases to focus less on the overall structure of data, and more on factors such as scalability and
performance~\cite{leavitt2010will}.

While the relational model is a good fit for many data problems, its strictness in terms of flexibility of managing data
eventually led to the introduction of the NoSQL model. The following characteristics can be given as a starting point
for NoSQL databases in comparison to relational databases~\cite{indrawan2012database}:

\begin{itemize}
  \item \textbf{Unstructured data support:} While the relational model would often force data to be stored in tabular
  formats, the NoSQL model does not force any kind of data schema.
  \item \textbf{Designed with distributed processing and horizontal scalability in mind:} Given the commoditisation of
  computer hardware in the last decade, support for horizontal scaling and processing among clusters is an important
  factor for adoption.
  \item \textbf{Less strict adherence to ACID principles:} While the relational model attempted to very much adhere to
  the transactional principles of data atomicty, consistency, isolation, and durability (ACID), this very much impacts
  performance in terms of distributed computing. Relaxing the strictness of adherence to these principles, allow many
  NoSQL databases to make the trade-off for higher performance.
\end{itemize}

Of course, these differences between the NoSQL model and relational model vary between each individual database
technology's design, and trade-offs are often made depending on the goals and aims for that given database.

In this paper, we will look at the use of MongoDB~\footnote{https://www.mongodb.org/}, a popular NoSQL database solution,
as a solution for a case study based upon a railway data problem using data from Monash University's Institute of Railway
Technology (IRT). An overview of the case study in question will be given in~\sectref{sec:case_study_overview}. A small
overview of the research already in this area will be given in~\sectref{sec:research_context}. Implementation and
evaluation details will be given in~\sectref{sec:implementation} and~\sectref{sec:evaluation_and_discussion}, respectively,
before concluding in~\sectref{sec:conclusion}.

% section introduction (end)

\newpage

\section{Case Study Overview} % (fold)
\label{sec:case_study_overview}

% section case_study_overview (end)

\newpage

\section{Research Context} % (fold)
\label{sec:research_context}

% section research_context (end)

\newpage

\section{Implementation} % (fold)
\label{sec:implementation}

% section implementation (end)

\newpage

\section{Evaluation and Discussion} % (fold)
\label{sec:evaluation_and_discussion}

% section evaluation_and_discussion (end)

\newpage

\section{Conclusion} % (fold)
\label{sec:conclusion}

% section conclusion (end)

\newpage                            % Make references section

\bibliographystyle{acm}
\bibliography{biblio.tex}

\end{document}
